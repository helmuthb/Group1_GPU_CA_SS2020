% !TeX spellcheck = en_US
\documentclass[sigconf,nonacm]{acmart}
\usepackage{listings}
\usepackage{caption}
\usepackage{color}
\newcounter{nalg}
\renewcommand{\thenalg}{\arabic{nalg}}
\DeclareCaptionLabelFormat{algocaption}{Algorithm \thenalg}

\lstnewenvironment{algorithm}[1][]
{   
    \refstepcounter{nalg}
    \captionsetup{labelformat=algocaption,labelsep=colon}
    \lstset{
        mathescape=true,
        frame=tB,
        numbers=left, 
        numberstyle=\tiny,
        basicstyle=\scriptsize, 
        keywordstyle=\color{black}\bfseries\em,
        keywords={,input, output, return, datatype, function, in, true, false, if, then, else, for, to, all, do, foreach, while, begin, end, }
        numbers=left,
        xleftmargin=.04\textwidth,
        #1 
    }
}
{}
\settopmatter{printacmref=false}
\pagestyle{empty}
\AtBeginDocument{%
  \providecommand\BibTeX{{%
    \normalfont B\kern-0.5em{\scshape i\kern-0.25em b}\kern-0.8em\TeX}}}
\copyrightyear{2020}
\acmYear{2020}
\setcopyright{rightsretained}
\begin{document}
\title{Group 1: Distributed Minimum Spanning Tree}
\subtitle{Prim's algorithm in C++, CUDA, CUDA Streaming and Thrust}
\author{Christian Kastner}
\email{christian.kastner@student.tuwien.ac.at}
\affiliation{Mat.Nr. 00100168}
\author{Adrian Tobisch}
\email{e1227508@student.tuwien.ac.at}
\affiliation{Mat.Nr. 01227508}
\author{Helmuth Breitenfellner}
\email{helmuth.breitenfellner@student.tuwien.ac.at}
\affiliation{Mat.Nr. 08725866}
\begin{abstract}
In this task we have been working on efficient implementations of
parallelized versions of Prim's algorithm (1957, \cite{prim1957}) for finding the Minimum Spanning Tree.
As a benchmark a sequential version, writting in plain C++, was compared.
\end{abstract}
\keywords{Prim, Minimum Spanning Tree, CUDA, C++, Thrust}
\maketitle
\section{Task description}

In this project we have implemented the following steps:
\begin{enumerate}
\item Base C++ classes for input/output
\item Graph generator, written in C++
\item Sequential version of Prim's algorithm
\item CUDA version of Prim's algorithm
\item Thrust implementation of Prim's algorithm
\item CUDA streaming version of Prim's algorithm
\item Runtime performance analysis
\end{enumerate}

\subsection{Base C++ classes}

We started with implementing a base C++ class \texttt{Graph},
which encapsulates functions like file input/output, counting the edges,
and storing the graph data.
It also has a flag specifying whether the graph shall be directed or
undirected.

This base class was used as the interface for all further implementations
of Prim's algorithm.

We also tried to consistently add unit tests to all functionality,
based on the \texttt{doctest} mini-framework.

\subsection{Graph generator}

The graph generator is creating a graph, with provided size (number of vertices),
density (approximate ratio of existing vs. possible edges), and the weight range.
Also a flag can be specified whether the graph shall be directed or undirected.

\subsection{Sequential version}

The sequential version is implementing the following pseudo code:
\begin{algorithm}[caption={Sequential Prim}, label={prim:cpu}]
MST-Prim(G, r):
for all v $\in$ G.V do
  in_MST[v] $\gets$ false
  distance[v] $\gets \infty$
  predecessor[v] $\gets$ null
end for
result_MST $\gets \emptyset$
in_MST[r] $\gets$ true
distance[r] $\gets$ 0
for all v $\in$ G.adj[r] do
  distance[v] $\gets$ G.weight[v,r]
  predecessor[v] $\gets$ r
end for
for i $\gets$ 1 to n-1 do
  v_next $\gets$ nearest_vertex(G, distance, in_MST)
  in_MST[v_next] $\gets$ true
  if predecessor[v_next] $\ne$ null then
    add edge(predecessor[v_next], v_next) to result_MST
  end if
  for all v $\in$ G.adj[v_next] do
    if v $\notin$ resultMST then
      if distance[v] > G.weight[v_next, v] then
        distance[v] $\gets$ G.weight[v_next, v]
        predecessor[v] $\gets$ v_next
      end if
    end if
  end for
end for
\end{algorithm}

\subsection{CUDA version of Prim's algorithm}
\subsection{Thrust implementation of Prim's algorithm}
\subsection{CUDA streaming version of Prim's algorithm}
\subsection{Runtime performance analysis}

\bibliographystyle{ACM-Reference-Format}
\bibliography{report}
\end{document}
